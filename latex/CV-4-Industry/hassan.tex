%%%%%%%%%%%%%%%%%
% This is an example CV created using altacv.cls (v1.1.5, 1 December 2018) written by
% LianTze Lim (liantze@gmail.com), based on the
% Cv created by BusinessInsider at http://www.businessinsider.my/a-sample-resume-for-marissa-mayer-2016-7/?r=US&IR=T
%
%% It may be distributed and/or modified under the
%% conditions of the LaTeX Project Public License, either version 1.3
%% of this license or (at your option) any later version.
%% The latest version of this license is in
%%    http://www.latex-project.org/lppl.txt
%% and version 1.3 or later is part of all distributions of LaTeX
%% version 2003/12/01 or later.
%%%%%%%%%%%%%%%%

%% If you are using \orcid or academicons
%% icons, make sure you have the academicons
%% option here, and compile with XeLaTeX
%% or LuaLaTeX.
% \documentclass[10pt,a4paper,academicons]{altacv}

%% Use the "normalphoto" option if you want a normal photo instead of cropped to a circle
% \documentclass[10pt,a4paper,normalphoto]{altacv}

\documentclass[10pt,a4paper,ragged2e]{altacv}

%% AltaCV uses the fontawesome and academicon fonts
%% and packages.
%% See texdoc.net/pkg/fontawecome and http://texdoc.net/pkg/academicons for full list of symbols. You MUST compile with XeLaTeX or LuaLaTeX if you want to use academicons.

% Change the page layout if you need to
\geometry{left=1.5cm,right=10.5cm,marginparwidth=8cm,marginparsep=1.2cm,top=1.25cm,bottom=1.25cm}
\usepackage{biblatex}

% Change the font if you want to, depending on whether
% you're using pdflatex or xelatex/lualatex
\ifxetexorluatex
  % If using xelatex or lualatex:
  \setmainfont{Carlito}
\else
  % If using pdflatex:
  \usepackage[utf8]{inputenc}
  \usepackage[T1]{fontenc}
  \usepackage[default]{lato}
\fi

\usepackage{hyperref}
\hypersetup{
    colorlinks=true,
    linkcolor=cyan,
    filecolor=magenta,      
    urlcolor=blue,
}
\usepackage{url}
% Change the colours if you want to
\definecolor{VividPurple}{HTML}{000000}
\definecolor{SlateGrey}{HTML}{2E2E2E}
\definecolor{LightGrey}{HTML}{2E2E2E}
\colorlet{heading}{VividPurple}
\colorlet{accent}{VividPurple}
\colorlet{emphasis}{SlateGrey}
\colorlet{body}{LightGrey}

% Change the bullets for itemize and rating marker
% for \cvskill if you want to
\renewcommand{\itemmarker}{{\small\textbullet}}
\renewcommand{\ratingmarker}{\faCircle}

%% sample.bib contains your publications
\addbibresource{../../references.bib}

\begin{document}
\name{Dr. Hassan ISMAIL FAWAZ}
\tagline{Senior Machine Learning Engineer | PhD in Artificial Intelligence}
% Cropped to square from https://en.wikipedia.org/wiki/Marissa_Mayer#/media/File:Marissa_Mayer_May_2014_(cropped).jpg, CC-BY 2.0
\photo{2.7cm}{photo.png}
\personalinfo{%
  % Not all of these are required!
  % You can add your own with \printinfo{symbol}{detail}
  \email{hassanismailfawaz@gmail.com}
%   \phone{+33769186977}
%  \mailaddress{Address, Street, 00000 County}
  \location{Saudi Arabia}
%  \homepage{marissamayr.tumblr.com/}
%  \twitter{@marissamayer}
 \github{\href{https://github.com/hfawaz}{hfawaz}}
%   \linkedin{https://www.linkedin.com/in/h-fawaz/}
%\homepage{\href{https://hfawaz.github.io}{hfawaz.github.io}}
% 
\faGraduationCap~~\href{https://scholar.google.com/citations?user=oUrGNaoAAAAJ&hl=en}{publications}
   % I'm just making this up though.
%   \orcid{https://orcid.org/0000-0003-1384-5996} % Obviously making this up too. If you want to use this field (and also other academicons symbols), add "academicons" option to \documentclass{altacv}
}

%% Make the header extend all the way to the right, if you want.
\begin{fullwidth}
\makecvheader
\end{fullwidth}

%% Depending on your tastes, you may want to make fonts of itemize environments slightly smaller
\AtBeginEnvironment{itemize}{\small}
% \begin{fullwidth}

\cvsection[page1sidebar]{Experiences}

\cvevent{Generative AI (GenAI) Expert}{GOSI (General Organization for Social Insurance)}{Feb 2024 -- Present}     {Riyadh, Saudi Arabia}

\begin{itemize}
	\item Developing GenAI applications to make \textbf{GOSI AI-First}
	\item Deploying and scaling \textbf{LLMs} in production on \textbf{GCP}
	\item Integrating APIs with \textbf{AI Agents} and Model Context Protocol
	
\end{itemize}

\divider 

\cvevent{Senior Machine Learning Engineer}{Beyond Limits}{Feb 2024 -- Feb 2025}     {Al Khobar, Saudi Arabia}

\begin{itemize}
	\item Scaling up and \textbf{load balancing} Large Language Models (\textbf{LLMs})
	\item \textbf{Project Lead} at the client's site for a large oil and gas operator
	\item \textbf{Data} pipelines for \textbf{completion} and \textbf{instruction tuning} 
	\item Maintain and develop LLMs in production (\textbf{GenAI} / LLMOps)
	\item Benchmark and evaluate various \textbf{vector databases}
	\item Developing advanced \textbf{Retrieval Augmented Generation}
	\item Fine tune \textbf{embedding} and \textbf{LLMs} (e.g. LORA)

\end{itemize}

\divider 

\cvevent{Senior Machine Learning Engineer}{Ericsson}{May 2022 -- Jan 2024}     {Paris, France}
\begin{itemize}
	\item \textbf{Data Architect}: Scaling \textbf{MLOps} framework for deep learning
	\vspace{0.3em}
	\begin{itemize}
		\item Build on top of managed Elastic \textbf{K8s} in \textbf{AWS}
		\item Setup \textbf{CI/CD} pipelines for infrastructure as code (\textbf{IAAC}) 
		\item Implement \textbf{experiment tracking} and serving using \textbf{MLFlow}
		\item Ensure \textbf{reproducibility} via \textbf{Docker} and \textbf{Kubeflow}
	\end{itemize}
    \item \textbf{Tech Lead}: In-house research \href{https://github.com/EricssonResearch/UDA-4-TSC}{project} on \textbf{transfer learning}
    \vspace{0.3em}
    \begin{itemize}
    	\item[-] Setup \textbf{best practices} for \textbf{Deep Learning} experimentation
        \item[-] Code review \& \textbf{mentoring} of PhD students \& data scientists
        \item[-] Implement state of the art models for \textbf{domain adaptation}
        \item[-] \textbf{Scale up Deep Learning experiments} on \textbf{Kubernetes}
        \item[-] Define the research project \textbf{roadmap} \& team responsibilities
        \item[-] Lead a monthly \textbf{reading group} on Artificial Intelligence
        \item[-] Animate training \textbf{workshops} for software best practices
    \end{itemize}
    \item \textbf{Machine Learning Engineer}: developing a microservice
    \vspace{0.3em}
    \begin{itemize}
        \item[-] Follow \textbf{software engineering} best practices
        \item[-] Interact with \textbf{Kafka} based communication system 
        \item[-] Develop \textbf{ML Ops} \& model \textbf{Life Cycle Management} (LCM)
    \end{itemize}
	
	\item \textbf{Data Engineer}: develop big data processing pipelines 
	\vspace{0.3em}
	\begin{itemize}
		\item[-] \textbf{Spark} for extracting \& normalizing \textbf{parquet} data
		\item[-] \textbf{Clustering} \& \textbf{Forecasting} of \textbf{time series} 
	\end{itemize}
	
\end{itemize}

\divider

\cvevent{Machine Learning Engineer}{Besedo}{Oct 2020 -- April 2022}     {Paris, France}
\begin{itemize}
	\item[-]  Content moderation using Machine Learning
	\item[-] Benchmark latest \textbf{Computer Vision} \textbf{Deep Learning} models
	\item[-] \textbf{Guide \& mentor} junior data scientists and linguistics
	\item[-] Evaluate \textbf{NLP} \textbf{transformer} models for \textbf{text classification}  
	\item[-] Develop \textbf{microservices} for live computer vision inference 
	\item[-] \textbf{Reduce latency} using quantization, pruning, JIT \& tensorRT
	\item[-] \textbf{Monitor} model \textbf{performance} and decay (e.g. accuracy)
	\item[-] Evaluate concept \& data \textbf{drift}, covariate \& label \textbf{shift}
	\item[-] Contribute to \textbf{HuggingFace}'s open source datasets (\href{https://github.com/pulls?q=is%3Apr+author%3Ahfawaz+archived%3Afalse+is%3Aclosed+huggingface+is%3Amerged}{link})
	\item[-] Deliver \textbf{proof-of-concepts} for ML models using \textbf{Streamlit}
\end{itemize}

\divider


\cvevent{Visiting Machine Learning Researcher}{Monash University}{Nov 2019 -- Dec 2019}     {Melbourne, Australia}
{Classifying satellite image time series.}\\
\divider

\cvevent{Deep Learning Lecturer}{Université Haute-Alsace}{Sep 2018 -- 2021}     {Mulhouse, France}
{Giving an advanced course on  deep neural networks.}\\

\divider

\cvevent{Software Developer Intern}{Orange Labs}{Mar 2017 -- Sep 2017}     {Nice, France}
\textbf{Java} recommendation system based on \textbf{RDF triplet data}

\divider

\cvevent{Freelance Web Development}{MradMCC}{Mar 2016 -- Aug 2016}{Beirut, Lebanon}
{Creating a WordPress website that can be found \href{http://www.mradmcc.com/home/}{here}.}

\divider

\cvsection{Journal Publications}
\printbibliography[heading=pubtype,title={\printinfo{\faFileTextO}{Journal Articles}}, type=article]




%\nocite{*}
%\citation{Meyer2000}
%\fullcite{Meyer2000}
%\citation{Meyer2000}
%\printbibliography

% \cvevent{Machine Learning \& Semantic Web Intern}{Orange Labs}{Mar 2017 -- Sep 2017}     {Sophia Antipolis, France}
% {Designing a dataset recommendation engine.}\\
% \divider

% \cvevent{Data Privacy \& Optimization}{Université Antonine}{Jun 2016 -- Aug 2016}  {Beirut, Lebanon}
% {Developing data anonymization technique with CPLEX.}\\
% \divider

%% Provide the file name containing the sidebar contents as an optional parameter to \cvsection.
%% You can always just use \marginpar{...} if you do
%% not need to align the top of the contents to any
%% \cvsection title in the "main" bar.


% \cvevent{Université de Bourgogne}{Masters in Databases \& Artificial Intelligence}{Sep 2016 -- Sep 2017}{Dijon, France}
% % \hbox {Rank= 2/30}
% \divider

% \cvevent{Université Antonine}{Masters in Software Engineering}{Sep 2012 -- Sep 2017}{Beirut, Lebanon}
% % \hbox {Perctange= 89.76\%}
% \divider



% \divider
%\cvskill{German}{3}

% \begin{itemize}
% \item Joined the company as employe \#20 and female employee \#1
% \item Developed targeted advertisement in order to use user's search queries and show them related ads
% \end{itemize}

%\cvsection{A Day of My Life}

% Adapted from @Jake's answer from http://tex.stackexchange.com/a/82729/226
% \wheelchart{outer radius}{inner radius}{
% comma-separated list of value/text width/color/detail}
% Some ad-hoc tweaking to adjust the labels so that they don't overlap
% \wheelchart{1.5cm}{0.5cm}{%
%   10/10em/accent!30/Sleeping \& dreaming about work,
%   25/9em/accent!60/Public resolving issues with Yahoo!\ investors,
%   5/13em/accent!10/\footnotesize\\[1ex]New York \& San Francisco Ballet Jawbone board member,
%   20/15em/accent!40/Spending time with family,
%   5/8em/accent!20/\footnotesize Business development for Yahoo!\ after the Verizon acquisition,
%   30/9em/accent/Showing Yahoo!\ employees that their work has meaning,
%   5/8em/accent!20/Baking cupcakes
% }

\clearpage

%\cvsection[page2sidebar]{Publications}
%
\nocite{*}
%
%\printbibliography[heading=pubtype,title={\printinfo{\faBook}{Books}},type=book]
%
%\divider
%
%\printbibliography[heading=pubtype,title={\printinfo{\faFileTextO}{Journal Articles}}, type=article]
%
%\divider
%
%\printbibliography[heading=pubtype,title={\printinfo{\faGroup}{Conference Proceedings}},type=inproceedings]


% %% If the NEXT page doesn't start with a \cvsection but you'd
% %% still like to add a sidebar, then use this command on THIS
% %% page to add it. The optional argument lets you pull up the
% %% sidebar a bit so that it looks aligned with the top of the
% %% main column.
% % \addnextpagesidebar[-1ex]{page3sidebar}


\end{document}
